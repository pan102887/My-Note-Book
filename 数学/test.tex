% !TeX root = test.tex
\RequirePackage{fix-cm}
\RequirePackage{amsmath,mathrsfs,amsfonts,amssymb}
\documentclass[a4paper, 12pt, draft]{article}
\usepackage[UTF8]{ctex}
\usepackage{amsmath}
\usepackage{mathabx}
\usepackage{amsfonts}


\title{基本初等函数的相关公式及其推导过程}
\author{}
\date{}


\begin{document}
\maketitle
\tableofcontents

\begin{abstract}
    本文主要介绍了基本初等函数的一些常用的公式,以及其推导过程。
\end{abstract}

\newpage
\section{对数函数}
    \subsection{和公式}
    \begin{equation}\label{eq1.1}
        \log_a{b} + \log_a{c} = \log_a{bc}
    \end{equation}
    推导过程:

    $$
    \begin{aligned}
        \because   b  &= a^{\log_{a}{b}}, c = a^{\log_{a}{c}}   \\
        \therefore bc &= a^{\log_{a}{b}} a^{\log_{a}{c}}        \\
                      &= a^{(\log_{a}{b} + \log_{a}{c})}        \\
                                                                &\\
        \therefore \log_{a}{bc} &= \log_{a}{a^{(\log_{a}{b} + \log_{a}{c})}} \\
                                &= \log_{a}{b} + \log_{a}{c}
    \end{aligned}
    $$

    \subsection{差公式}
    \begin{equation}\label{eq1.2}
        \log_{a}{b} - \log_{a}{c} = \log_{a}{ \dfrac{b}{c} }
    \end{equation}
    
    推导过程:
    $$
    \begin{aligned}
        \because   b            &= a^{\log_{a}{b}}, c = a^{\log_{a}{c}}     \\
        \therefore \dfrac{b}{c}  &= \dfrac{a^{\log_{a}{b}}}{a^{\log_{a}{c}}}  \\
                                &= a^{(\log_{a}{b} + \log_{a}{c})}          \\
                                &                                           \\
        \therefore \log_{a}{\dfrac{b}{c}} &= \log_{ a }{ a ^ {(\log_{a}{b}-\log_{a}{c})} } \\
                                &= \log_{a}{b} - \log_{a}{c}
    \end{aligned}
    $$

    \newpage
    \subsection{换底公式}
    \begin{equation}\label{eq1.3}
        \log_{a}{b} = \dfrac{ \log_{n}{a} }{ \log_{n}{b} }
    \end{equation}
    推导过程:
    $$
    \begin{aligned}
        \because    \log_{n}{b} &= \log_{n}{a} \log_{a}{b} \\
        \therefore  \log_{a}{b} &= \dfrac{ \log_{n}{b} }{ \log_{n}{a} }
    \end{aligned}
    $$


    \subsection{真底互换公式}
    \begin{equation}\label{eq1.4}
        \log_{a}{b} = \dfrac{1}{ \log_{b}{a} }
    \end{equation}
    推导过程:
    $$
    \begin{aligned}
        \because    \log_{a}{a} &= \log_{a}{b} \log_{b}{a} = 1 \\
        \therefore  \log_{a}{b} &= \dfrac{1}{ \log_{b}{a} }
    \end{aligned}
    $$


    \subsection[未知公式]{不知道是什么公式}
    \begin{equation}\label{1.5}
        \log_{a}{b} = \dfrac{ \log_{a}{c} }{ \log_{b}{c} }
    \end{equation}
    推导过程:
    $$
    \begin{aligned}
        & \because \log_{a}{c} = \log_{a}{b^{ \log_{b}{c} }} = \log_{b}{c} \log_{a}{b} \\
        & \therefore \log_{a}{b} = \dfrac{ \log_{a}{c} }{ \log_{b}{c} }
    \end{aligned}
    $$

    \newpage
    \section{三角函数}
    \subsection{诱导公式}

    \paragraph{公式一}
    \begin{equation}
        \begin{aligned}
            &\sin{(\alpha + 2k\pi)} = \sin{\alpha} \qquad (k \in \mathbb{Z} ) \\
            &\cos{(\alpha + 2k\pi)} = \cos{\alpha} \qquad (k \in \mathbb{Z} ) \\
            &\tan{(\alpha + 2k\pi)} = \tan{\alpha} \qquad (k \in \mathbb{Z} )
        \end{aligned}
    \end{equation}

    \paragraph{公式二}
    \begin{equation}
        \begin{aligned}
            &\sin{(\alpha + \pi)} = -\sin{\alpha} \\
            &\cos{(\alpha + \pi)} = -\cos{\alpha} \\
            &\tan{(\alpha + \pi)} =  \tan{\alpha} \\
            &\cot{(\alpha + \pi)} =  \cot{\alpha}
        \end{aligned}
    \end{equation}

    \paragraph{公式三}
    \begin{equation}
        \begin{aligned}
            &\sin{(-\alpha)} = -\sin{\alpha} \\
            &\cos{(-\alpha)} =  \cos{\alpha} \\
            &\tan{(-\alpha)} =  -\tan{\alpha} \\
            &\cot{(-\alpha)} =  -\cot{\alpha}
        \end{aligned}
    \end{equation}

    \paragraph{公式四}
    \begin{equation}
        \begin{aligned}
            &\sin{(\pi - \alpha)} = \sin{\alpha} \\
            &\cos{(\pi - \alpha)} = -\cos{\alpha} \\
            &\tan{(\pi - \alpha)} = -\tan{\alpha} \\
            &\cot{(\pi - \alpha)} = -\cot{\alpha}
        \end{aligned}
    \end{equation}

    \paragraph{公式五}
    \begin{equation}
        \begin{aligned}
            &\sin{(\alpha + \dfrac{k\pi}{2})} = \sin{\alpha} \\
            &\cos{(\alpha + \dfrac{k\pi}{2})} = -\cos{\alpha} \\
            &\tan{(\alpha + \dfrac{k\pi}{2})} = -\tan{\alpha} \\
            &\cot{(\alpha + \dfrac{k\pi}{2})} = -\cot{\alpha}
        \end{aligned}
    \end{equation}


    \subsection{二角和差公式}
    \begin{equation}\label{二角和差公式}
        \cos{(\alpha - \beta)} = \sin{\alpha}\sin{\beta} + \cos{\alpha}\cos{\beta}
    \end{equation}

    推导过程如下:
    设与x轴的夹角分别为$\alpha$和$\beta$的两个单位向量为$\vec{A}$和$\vec{B}$,
    则有:
    $$
    \begin{aligned}
        &\vec{A} = (\sin{\alpha}, \cos{\alpha}) \quad \vec{B} = (\sin{\beta}, \cos{\beta})     \\
        \therefore &\vec{A} \cdot \vec{B} = \sin{\alpha}\sin{\beta} + \cos{\alpha}\cos{\beta}  \\
        \therefore &\vec{A} \cdot \vec{B} = |\vec{A}||\vec{B}|\cos{(\alpha - \beta)}           \\
        \therefore &\sin{\alpha}\sin{\beta} + \cos{\alpha}\cos{\beta} = \cos{(\alpha - \beta)} \\
    \end{aligned}
    $$
    将$\beta = -\beta$带入上式,则有
    \begin{equation}\label{2.3}
        \cos{(\alpha + \beta)} = \cos{\alpha}\cos{\beta} - \sin{\alpha}\sin{\beta}
    \end{equation}
    再将$\beta = \beta - \dfrac{\pi}{2}$带入上式,则有
    $$
    \cos{(\alpha + \beta - \dfrac{\pi}{2})} = \cos{\alpha}\cos{(\beta - \dfrac{\pi}{2})} - \sin{\alpha}\sin{(\beta - \dfrac{\pi}{2})}
    $$
    因此可得
    \begin{equation}\label{2.4}
        \sin{(\alpha + \beta)} = \cos{\alpha}\sin{\beta} + \sin{\alpha}\cos(\beta)
    \end{equation}
    同理,将$\beta = -\beta$带入上式,则有:
    $$
    \sin{(\alpha - \beta)} = \cos{\alpha}\sin{(-\beta)}+\sin{\alpha}\cos{(-\beta)}
    $$
    易得
    \begin{equation}\label{2.5}
        \sin{(\alpha - \beta)} = \sin{\alpha}\cos{\beta} - \cos{\alpha}\sin{\beta}
    \end{equation}
    
\end{document}
