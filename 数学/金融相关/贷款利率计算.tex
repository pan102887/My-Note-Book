% !TeX root = 贷款利率计算.tex
\RequirePackage{fix-cm}
\RequirePackage{amsmath,mathrsfs,amsfonts,amssymb}
\documentclass[a4paper, 12pt, draft]{article}
\usepackage[UTF8]{ctex}
\usepackage{amsmath}
\usepackage{mathabx}
\usepackage{amsfonts}

\title{各种贷款还款方式利息计算}
\author{}
\date{}

\begin{document}
\maketitle
\tableofcontents
\section{等额本息}


等额本息的还款方式特点是每一期的还款金额都相等。
设每期还款额为x,本金为m,
每期利率为r(如果还款周期是每个月的话,这个通常是年利率/12),
总期数为n,则有:

$$ x = m \dfrac{ r (1+r)^n }{ (1+r)^n - 1} $$

\newpage
\subsection[利息公式推导]{等额本息还款方式利息公式推导}
其推导过程可以设$C_n$为第n期还款后,仍欠银行的金额,则有:


\begin{equation}
    \begin{aligned}
        C_1 & = m (1+r) - x                                             \\
        \\
        C_2 & = C_1 (1+r) - x                                           \\
            & = [m (1+r) - x] (1+r) - x                                 \\
            & = m (1+r)^2 - x [1 + (1+r)^1]                             \\
            & = m (1+r)^2 - x [(1+r)^0 + (1+r)^1]                       \\
        \\
        C_3 & = C_2 (1+r) - x                                           \\
            & = [ m (1+r)^2 - x [(1+r)^0 + (1+r)^1] ] (1+r) - x         \\
            & = m (1+r)^3 - x [ (1+r)^0 + (1+r)^1 + (1+r)^2 ]           \\
        \\
            & \vdots                                                    \\
        C_n & = m (1+r)^n - x \sum_{ i=0 }^{ n-1 } (1+r)^i              \\
            & = m (1+r)^n - x \dfrac{ (1+r)^n - 1 }{ r }                \\
        \\
            & \because \space 还完第N期还款后,仍欠银行的金额为0,即C_n = 0               \\
            & \therefore m (1 + r)^n - x \dfrac{ (1+r)^n - 1 }{ r } = 0 \\
            & \therefore x = m \dfrac{ r (1+r)^n }{ (1+r)^n - 1 }       \\
        \\
    \end{aligned}
\end{equation}


\newpage
\subsection{利息的计算}
由于每期固定还款额度为$x$,还款期数为$n$,本金为$m$,因此总利息可以使用$xn-m$表示,设函数$t(m, r, n)$表示总利息,有:

\begin{equation}
    \begin{aligned}
        t(m, r, n) & = m n \dfrac{ r (1 + r)^n }{ (1 + r)^n - 1 } - m     \\
                   & = m [ n \dfrac{ r (1 + r)^n }{ (1 + r)^n - 1 } - 1 ]
    \end{aligned}
\end{equation}

\subsection{TODO LIST}
\subsubsection{使用Python实现利息计算器}
可以看到,总利息的大小与本金,利率,以及还款期数有关。并且上文给出了计算公式。
现在根据计算公式,使用Python,写一个根据本金,利率,还款周期计算利息的方法。
要求精度保留小数点后8位。



\end{document}