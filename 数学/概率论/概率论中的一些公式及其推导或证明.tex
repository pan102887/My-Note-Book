% !Tex root = 概率论中的一些公式及其推导或证明.tex

\RequirePackage{fix-cm,amsmath,mathrsfs,amsfonts,amssymb}
\documentclass[a4paper, 12pt, draft]{article}
\usepackage[UTF8]{ctex}
\usepackage{amsmath}
\usepackage{mathabx}
\usepackage{amsfonts}

\title{概率论中的一些公式及其推导或证明}
\author{}
\date{}

\begin{document}
\maketitle


    \section{组合数的计算} 

    \begin{equation*}
        \begin{aligned}
            C_{n}^{m} &= \dfrac{n(n-1) \cdots (n - m + 1)}{m!} \\
                      &= \dfrac{A_n^m}{m!} = \dfrac{n!}{m!(n-m)!}
        \end{aligned}
    \end{equation*}


    \section{二项分布}
    在n次独立重复试验(伯努利试验)中,事件A发生的次数X服从二项分布,记为$X \sim B(n, p)$,其概率分布为:
    
    \begin{equation*}
    P(X=k) = C_n^k p^k (1-p)^{n-k}
    \end{equation*}


    \section{泊松分布}
    在n重伯努利试验中,记A事件在一次试验中发生的概率为$p_n$,如果当$n \to + \infty$时,有$np_n \to \lambda(>0)$,则

    \begin{equation*}
    \lim_{n \to + \infty} \binom{n}{k} p_n^k (1-p_n)^{n-k} = \dfrac{\lambda^k}{k!} e^{-\lambda}
    \end{equation*}

    证明:
    设$np_{n} = \lambda_{n}$,则有$p_n=\dfrac{\lambda_{n}}{n}$

    \begin{equation*}
    \begin{aligned}
        \binom{n}{k} p_n^k (1-p_n)^{n-k} &= \dfrac{n(n-1) \cdots (n-k+1)}{k!} \left( \dfrac{\lambda_n}{n} \right)^{k} \left( 1 - \dfrac{\lambda_n}{n} \right)^{n-k} \\
        &= \dfrac{\lambda_n^{k}}{k!} \left( \dfrac{n}{n}\right)\left( 1-\dfrac{1}{n} \right) \cdots \left( 1-\dfrac{k-1}{n} \right)\left( 1 - \dfrac{\lambda_n}{n} \right)^{n-k} .
    \end{aligned}
    \end{equation*}

    对于固定的K,有

    \begin{equation*}
    \begin{aligned}
        \lim_{n \to + \infty} \lambda_n = \lambda \\
        \lim_{n \to + \infty} \left(1 - \dfrac{\lambda_n}{n}\right)^{n-k} = e^{-\lambda}, \\
    \end{aligned}
    \end{equation*}

    因此有

    \begin{equation*}
    \lim_{n \to + \infty} \binom{n}{k} p_n^k (1-p_n)^{n-k} = \dfrac{\lambda^k}{k!} e^{-\lambda}
    \end{equation*}


\end{document}